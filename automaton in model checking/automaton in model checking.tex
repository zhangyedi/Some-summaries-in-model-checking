\documentclass{sciposter}
\usepackage[dvipsnames,usenames,svgnames,table]{xcolor}
\usepackage{lipsum}
\usepackage{epsfig}
\usepackage{amsmath}
\usepackage{amssymb}
\usepackage[german]{babel}
\usepackage{geometry}
\usepackage{multicol}
\usepackage{graphicx}
\usepackage{tikz}
\usepackage{wrapfig}
\usepackage{gensymb}
\usepackage[utf8]{inputenc}
\usepackage{empheq}
\usepackage[colorlinks,linkcolor=red]{hyperref}


\geometry{
 landscape,
 a1paper,
 left=5mm,
 right=50mm,
 top=5mm,
 bottom=50mm,
 }



%BEGIN LISTINGDEF



\usepackage{listings}
\usepackage{sourcecodepro}
\definecolor{listing-background}{rgb}{0.97,0.97,0.97}
\definecolor{listing-rule}{HTML}{B3B2B3}
\definecolor{listing-numbers}{HTML}{B3B2B3}
\definecolor{listing-text-color}{HTML}{000000}
\definecolor{listing-keyword}{HTML}{435489}
\definecolor{listing-identifier}{HTML}{435489}
\definecolor{listing-string}{HTML}{00999a}
\definecolor{listing-comment}{HTML}{8e8e8e}
\definecolor{listing-javadoc-comment}{HTML}{006CA9}

\lstdefinestyle{eisvogellistingstyle}{
	language=java,
	numbers=left,
	backgroundcolor=\color{listing-background},
	basicstyle=\color{listing-text-color}\small\ttfamily{}, % print whole listing small
	xleftmargin=0.8em, % 2.8 with line numbers
	breaklines=true,
	frame=single,
	framesep=0.6mm,
	rulecolor=\color{listing-rule},
	frameround=ffff,
	framexleftmargin=0.4em, % 2.4 with line numbers | 0.4 without them
	tabsize=4, %width of tabs
	numberstyle=\color{listing-numbers},
	aboveskip=1.0em,
	keywordstyle=\color{listing-keyword}\bfseries, % underlined bold black keywords
	classoffset=0,
	sensitive=true,
	identifierstyle=\color{listing-identifier}, % nothing happens
	commentstyle=\color{listing-comment}, % white comments
	morecomment=[s][\color{listing-javadoc-comment}]{/**}{*/},
	stringstyle=\color{listing-string}, % typewriter type for strings
	showstringspaces=false, % no special string spaces
	escapeinside={/*@}{@*/}, % for comments
	literate=
	{á}{{\'a}}1 {é}{{\'e}}1 {í}{{\'i}}1 {ó}{{\'o}}1 {ú}{{\'u}}1
	{Á}{{\'A}}1 {É}{{\'E}}1 {Í}{{\'I}}1 {Ó}{{\'O}}1 {Ú}{{\'U}}1
	{à}{{\`a}}1 {è}{{\'e}}1 {ì}{{\`i}}1 {ò}{{\`o}}1 {ù}{{\`u}}1
	{À}{{\`A}}1 {È}{{\'E}}1 {Ì}{{\`I}}1 {Ò}{{\`O}}1 {Ù}{{\`U}}1
	{ä}{{\"a}}1 {ë}{{\"e}}1 {ï}{{\"i}}1 {ö}{{\"o}}1 {ü}{{\"u}}1
	{Ä}{{\"A}}1 {Ë}{{\"E}}1 {Ï}{{\"I}}1 {Ö}{{\"O}}1 {Ü}{{\"U}}1
	{â}{{\^a}}1 {ê}{{\^e}}1 {î}{{\^i}}1 {ô}{{\^o}}1 {û}{{\^u}}1
	{Â}{{\^A}}1 {Ê}{{\^E}}1 {Î}{{\^I}}1 {Ô}{{\^O}}1 {Û}{{\^U}}1
	{œ}{{\oe}}1 {Œ}{{\OE}}1 {æ}{{\ae}}1 {Æ}{{\AE}}1 {ß}{{\ss}}1
	{ç}{{\c c}}1 {Ç}{{\c C}}1 {ø}{{\o}}1 {å}{{\r a}}1 {Å}{{\r A}}1
	{€}{{\EUR}}1 {£}{{\pounds}}1 {«}{{\guillemotleft}}1
	{»}{{\guillemotright}}1 {ñ}{{\~n}}1 {Ñ}{{\~N}}1 {¿}{{?`}}1
}
\lstset{style=eisvogellistingstyle}



%END LISTINGDEF


\newcommand*\widefbox[1]{\fbox{\hspace{2em}#1\hspace{2em}}}


\newlength\dlf  % Define a new measure, dlf
\newcommand\alignedbox[2]{
% Argument #1 = before & if there were no box (lhs)
% Argument #2 = after & if there were no box (rhs)
&  % Alignment sign of the line
{
\settowidth\dlf{$\displaystyle #1$}
    % The width of \dlf is the width of the lhs, with a displaystyle font
\addtolength\dlf{\fboxsep+\fboxrule}
    % Add to it the distance to the box, and the width of the line of the box
\hspace{-\dlf}
    % Move everything dlf units to the left, so that & #1 #2 is aligned under #1 & #2
\boxed{#1 #2}
    % Put a box around lhs and rhs
}
}
\usepackage{graphicx,url}

%BEGIN TITLE
\title{\huge{My summaries about automaton used in model checking}}

\author{\large{Zhang Yedi}}
%END TITLE

\begin{document}
\fontfamily{phv}\selectfont

\maketitle



\begin{multicols}{3}
\section{First Section}	
    \textbf{1. }One scenario where \textbf{\emph{'automaton is closed under complementation'}} is necessary:

    ~~~~If an automaton $\mathcal{A}$ is closed under complementation, and an AMA $\mathcal{M}^{\langle\langle A\rangle\rangle\varphi}$ can recognize the set of configurations of $\mathcal{P}$ satisfying $\langle\langle A \rangle\rangle\varphi$ in a bottom-up approach. We assume that $\mathcal{M}^{\neg\langle\langle A \rangle\rangle\varphi }$ has also been computed to recognize $\mathcal{C}_{\mathcal{P}}$ (T Chen et al. 2016).
    \\

    \textbf{2. Automaton's production} (\href{https://cs.stackexchange.com/questions/41268/product-of-a-transition-system-and-a-finite-automaton}{reference})
    :

    ~~~~The question is like:

    \includegraphics{figures/1-1-question.jpg}~\\[1cm]

    ~~~~And the answer should be like:

    \includegraphics{figures/1-1-answer.jpg}~\\[1cm]


    \textbf{3. Why PDA $\rightarrow$ Multi-Automaton is necessary?}

    ~~~~PDA has an infinite sets of states (because it has an infinite stack space, even though its control states and alphabet are all finite). So for a PDA, a state should be a configuration like $\langle p_i,\omega \rangle$. In the case of finite states systems, the sets $X_i$ are all finite, and the sequence $\{X_i\}_{i\geq 0}$ is guaranteed to reach a fix point, which immediately provides an algorithm to compute $pre^*(S)$. Unfortunately, these properties no longer hold for any non-trivial class of infinite states systems.

    \textbf{~~~~Introduction to Alternating Multi-Automaton:} We can regard that an AMA has one initial state for each control location in Pushdown automaton, so for AMA, it doesn't just has only one initial state, instead, it has a initial states set. And the automaton recognizes the configuration $\langle p,\omega\rangle$ if it accepts the word $\omega$ from the initial state corresponding to $p$. (AMA is just a tool to represent a set of configurations, and not to confuse its "behaviour" with that of the pushdown system.)\cite{BouajjaniEM97}.


    ~~~~\textbf{Multi-Automaton:} Given a MA $\mathcal{A}$, it returns a \emph{\textbf{regular}} set of configurations $C$ of PDA, what's more, we can get another MA $\mathcal{A}_{pre^*}$ recognizing $pre^*(C)$ (a closure set of $pre(C)$).

%Example Listing
\begin{lstlisting}[language=Python]
def main():
	return Null
\end{lstlisting}


\end{multicols}

\bibliographystyle{alpha}
\bibliography{references}
\end{document} 